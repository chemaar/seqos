%
% cabeceras.tex
%

%\usepackage[T1]{fontenc}

\definecolor{ZurichBlue}{rgb}{.255,.41,.884}

\beamertemplateshadingbackground{white!10}{white!10}

\usepackage{beamerthemeWarsaw}
\usepackage{longtable}

%\usecolortheme[named=OliveGreen]{structure} 
\setbeamertemplate{items}[ball] 
\setbeamertemplate{blocks}[rounded][shadow=true] 
\setbeamertemplate{footline}[page number]
\addtocounter{framenumber}{-1}
%Handout
%\usepackage{handoutWithNotes}
%\usepackage{tikz,times}
%\pgfpagesuselayout{2 on 1 with notes}[a4paper,border shrink=5mm]

\usepackage{beamerthemeshadow}
 \useoutertheme[hooks]{tree}
 
% \setbeamertemplate{headline}[default] % The default is just an empty headline.
% \setbeamertemplate{headline}[infolines theme]
% \setbeamertemplate{headline}[miniframes theme]
% \setbeamertemplate{headline}[sidebar theme]
% \setbeamertemplate{headline}[smoothtree theme]
% \setbeamertemplate{headline}[smoothbars theme]
% \setbeamertemplate{headline}[tree]
\beamertemplatetransparentcovereddynamic

% spanish
\usepackage[spanish]{babel}
\usepackage[utf8]{inputenc}

% diagramas
%\usepackage{pst-eps,epstopdf}
\usepackage{pst-node}
%\usepackage{pst-all}
\usepackage{pst-blur}
%\usepackage{pst-tree}

% incrustaciones de código fuente
\usepackage{listings}

% matemáticas y símbolos
\usepackage{amsmath}
\usepackage{amssymb}
\usepackage[right]{eurosym}
\usepackage{ulem}

% colores
\usepackage{colortbl}

%\usepackage{algorithm2e}
%\usepackage{algorithm}
%\usepackage{algorithmic}


\lstset{%
  language=Python,
	basicstyle=\footnotesize\sffamily,
	keywordstyle=\color{darkred},
 	stringstyle=\color{violet},
 	commentstyle=\color{blue},
 	showspaces=false,
 	showtabs=false,
 	showstringspaces=false,
 	frame=trBL,
        frameround=tttt,
       % backgroundcolor=\color{lightyellow},
 	extendedchars=true,
 	numbers=none,
        aboveskip=0.5cm,
        belowskip=0.5cm,
        xleftmargin=1cm,
        xrightmargin=1cm,
	breaklines=true
}
\definecolor{darkred}{rgb}{0.5, 0, 0}
\definecolor{violet}{rgb}{1, 0, 1}
\definecolor{lightyellow}{rgb}{1,1,0.8}


\usepackage{latexsym}
\usepackage{amsmath}
\usepackage{amssymb}
\usepackage{amsthm}

\usepackage{xspace}

\newcommand{\si}{$\oplus$\xspace}
\newcommand{\no}{$\ominus$\xspace}
\newcommand{\na}{$\odot$\xspace}

%Extraer
\newcommand{\linkeddata}{\textit{Linked Data}\xspace}
\newcommand{\opendata}{\textit{Open Data}\xspace}
\newcommand{\lod}{\textit{Linking Open Data}\xspace}
\newcommand{\ogd}{\textit{Open Government Data}\xspace}
\newcommand{\datasets}{\textit{datasets}\xspace}
\newcommand{\dataset}{\textit{dataset}\xspace}
\newcommand{\provenance}{\textit{provenance}\xspace}
\newcommand{\trust}{\textit{trust}\xspace}
\newcommand{\egov}{\textit{e-government}\xspace}
\newcommand{\pusi}{\textit{Public Sector Information}\xspace}
\newcommand{\gd}{\textit{Government Data}\xspace}
\newcommand{\wod}{Web de Datos\xspace}
\newcommand{\wode}{\textit{Web of Data}\xspace}
\newcommand{\eproc}{\textit{e-Procurement}\xspace}
\newcommand{\gld}{\textit{Government Linked Data}\xspace}

\hyphenation{real}

\newrgbcolor{ColorEncabezadoTabla}{0.7 0.7 0.9}
\newrgbcolor{ColorFila1}{0.8 0.8 0.7}
\newrgbcolor{ColorFila2}{0.8 0.7 0.8}
\newrgbcolor{ColorTotal}{0.7 0.9 0.7}


% \usepackage{tikz,times}
% \usetikzlibrary{mindmap,backgrounds}
